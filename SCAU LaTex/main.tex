%使用XeLaTex或者LualaLaTex编译

\documentclass[UTF8,AutoFakeBold=1,AutoFakeSlant,zihao=-4]{SCAU}
\usepackage{graphicx} % 用于插入图片
\usepackage{setspace} % 用于设置行距
\usepackage{ctex} % 支持中文字体和字号设置
\usepackage{color}
\definecolor{lightgray}{rgb}{.9,.9,.9}
\definecolor{darkgray}{rgb}{.4,.4,.4}
\definecolor{purple}{rgb}{0.65, 0.12, 0.82}

\lstdefinelanguage{JavaScript}{
  keywords={typeof, new, true, false, catch, function, return, null, catch, switch, var, if, in, while, do, else, case, break, const},
  keywordstyle=\color{blue}\bfseries,
  ndkeywords={class, export, boolean, throw, implements, import, this},
  ndkeywordstyle=\color{darkgray}\bfseries,
  identifierstyle=\color{black},
  sensitive=false,
  comment=[l]{//},
  morecomment=[s]{/*}{*/},
  commentstyle=\color{purple}\ttfamily,
  stringstyle=\color{red}\ttfamily,
  morestring=[b]',
  morestring=[b]"
}



%%%%%%%%%%%%%%%请在这里填写你的论文题目
\newcommand{\thesisTitle}{SCAU论文 \LaTeX 使用手册}

%%%%%%%%%%%%%%请在这里填写你的个人信息
\newcommand{\yourDept}{电子工程学院}  %学院
\newcommand{\yourMajor}{光电信息科学与工程}  %专业
\newcommand{\yourName}{残橘子}     %名字
\newcommand{\yourMentor}{张三}        %导师
\newcommand{\yourjob}{校长}           %职称
\newcommand{\studentID}{202200000000}   %学号
\newcommand{\Date}{2024年11月30日}         %日期,也可以使用\zhtoday代替,即中文的当天年月日





\begin{document}

% 封面(自动生成)
\coverpage      %%填写全部信息以确保no warning

%如果你需要印双面,请注意留白
%\newpage
%\thispagestyle{empty}  % 设置空白页不显示页眉页脚
%\mbox{}  % 生成一个空白页
%\newpage


\begin{authorization}
%此处为原创性声明,不用填写.也不用删除
\end{authorization}

% 中文摘要
\begin{abstract}
本文详细介绍了如何使用 LaTeX 排版华南农业大学(SCAU)的学术论文。内容涵盖了 LaTeX 的基本使用方法、论文格式要求(如标题、目录、正文、参考文献等),以及常见问题的解决方案。本文特别针对 SCAU 论文的格式需求,提供了详细的代码示例和配置说明,包括如何设置中文字体、调整段落格式、插入图表、管理参考文献等。通过本手册,用户可以快速掌握 LaTeX 的使用技巧,高效完成符合 SCAU 标准的学术论文排版。
\keywords{\LaTeX ;SCAU 论文;排版;参考文献;目录;插图}%填写中文关键词,使用全角分号隔开

\end{abstract}


\begin{center}
    {\textbf{\large{A Comprehensive LaTeX Guide for SCAU Thesis Formatting}}} \\[1ex]        %%%%请替换你的论文英文名
    HurtOrange \\[1ex]         %%%%请在此处修改你的英文名
    (College of Science, South China Agricultural University, Guangzhou 510642, China)      %%此处不用修改
\end{center}

\noindent{\textbf{Abstract:}        %%%%%此处修改你的英文摘要
This article provides a comprehensive guide on using LaTeX to typeset academic papers for South China Agricultural University (SCAU). It covers the basic usage of LaTeX, formatting requirements for SCAU theses (such as titles, table of contents, main text, and references), and solutions to common issues. Specifically tailored to the formatting needs of SCAU theses, this manual offers detailed code examples and configuration instructions, including how to set Chinese fonts, adjust paragraph formatting, insert figures and tables, and manage references. With this guide, users can quickly master LaTeX and efficiently produce academic papers that meet SCAU's standards.

\vspace{1cm}

\noindent \textbf{Key words:} \LaTeX ; SCAU Thesis; Typesetting; References; Table of Contents; Figures          %%%采用半角分号隔开
%%%%%%%%%此处修改你的关键词

	\addcontentsline{toc}{section}{Abstract}  
    %添加目录,在类文件里的英文摘要没有使用,这里重新添加目录
    
\newpage        %起新页面

% 英文摘要

%%这里不建议使用该部分做英文摘要,格式不正确。请使用上面的填充即可。因为我无法做出正确的英文摘要界面,所以用上面的代替.
%\begin{abstractEN}
%\keywordsEN
%\end{abstractEN}

% 目录(自动生成),不用修改
\contentpage



%%%%%%%%%%%%%%%%%%%以下是正文部分%%%%%%%%%%%%%%%%%%%%%%%

\section{LaTeX 历史}

\subsection{TeX 的诞生}
\subsubsection{Donald Knuth 的动机}
Donald Knuth 是 TeX 的创始人。他在 1970 年代编写《计算机程序设计艺术》时,对当时的排版工具感到不满,尤其是数学公式的排版质量。这促使他决定开发一种新的排版系统。

\subsubsection{TeX 的开发}
Knuth 于 1978 年开始开发 TeX,并在 1982 年发布了第一个稳定版本。TeX 的设计目标是实现高质量的排版,特别是在数学公式和复杂文档方面。

\subsection{LaTeX 的出现}
\subsubsection{Leslie Lamport 的贡献}
Leslie Lamport 在 1980 年代基于 TeX 开发了 LaTeX。LaTeX 提供了更高层次的抽象,使得用户可以专注于文档内容,而不必过多关注排版细节。

\subsubsection{LaTeX 的普及}
LaTeX 的易用性和强大功能使其迅速在学术界和科研领域普及,成为撰写学术论文、书籍和技术文档的首选工具。

\subsection{LaTeX 的发展}
\subsubsection{LaTeX2e 的发布}
1994 年,LaTeX2e 发布,成为 LaTeX 的当前标准版本。LaTeX2e 引入了许多新功能和改进,进一步提升了 LaTeX 的灵活性和易用性。



\paragraph{社区的支持}
LaTeX 的成功离不开全球用户和开发者的支持。许多宏包和工具被开发出来,扩展了 LaTeX 的功能,使其能够满足各种排版需求。

\paragraph{未来的方向}
随着技术的发展,LaTeX 仍在不断进化。新的工具和集成环境(如 Overleaf)使得 LaTeX 更加易于使用,同时也保持了其高质量排版的核心优势。


\section{图片示例}

在 LaTeX 中插入图片非常简单。以下是一个示例:

\begin{figure}[h] % 使用 [h] 强制图片位置
  \centering
  \includegraphics[width=0.5\textwidth]{Fig/latex.jpg} % 插入图片,宽度为文本宽度的一半
  \caption{这是一个示例图片}    % 图题
  \label{latex}          % 图片标签
  \figsource{资料来源:网络搜素}     %图注
\end{figure}

如图 \ref{latex} 所示,这是一个示例图片。



\section{表格示例}


以下是 \LaTeX 和 Microsoft Word 的优劣比较:

\begin{table}[h] % 使用 [h] 强制表格位置
  \centering
  \caption{\LaTeX  与 Word 优劣比较} % 表格标题
  \label{tab:latex-vs-word} % 表格标签
  \begin{tabular}{lcc} % 定义表格列格式(3 列,第一列左对齐,其余居中)
    \toprule % 顶部横线
    \textbf{比较项} & \textbf{\LaTeX} & \textbf{Word} \\
    \midrule % 中间横线
    排版质量 & 高 & 中 \\
    易用性 & 低 & 高 \\
    功能与灵活性 & 高 & 中 \\
    协作与兼容性 & 中 & 高 \\
    适用场景 & 学术论文、技术文档 & 日常办公、简单文档 \\
    \bottomrule % 底部横线
    
  \end{tabular}
\end{table}
      \tablesource{这是表注}

如表 \ref{tab:latex-vs-word} 所示,LaTeX 和 Word 在排版质量、易用性、功能与灵活性、协作与兼容性以及适用场景等方面各有优劣\cite{bai2002kuo}\cite{xu2024lun}。

\section{代码示例}

\subsection{Python代码示例}
\begin{lstlisting}[language=Python]
print("Hello, World!")
\end{lstlisting}

\subsection{汇编 (89C51)代码示例}
\begin{lstlisting}[language={[x86masm]Assembler}]
ORG 0H        ; 程序起始地址
MOV DPTR, #MSG ; 将 MSG 地址加载到数据指针寄存器中
MOV SBUF, #0   ; 清空 SBUF 寄存器

; 发送字符串
SEND_STRING:
    MOV A, @DPTR ; 将 DPTR 指向的数据加载到 A 寄存器
    INC DPTR     ; 数据指针寄存器自增,指向下一个字符
    JZ END       ; 如果 A 寄存器为 0(字符串结束符),则跳转到 END
    MOV SBUF, A  ; 将 A 寄存器中的字符送到 SBUF 寄存器
    JNB TI, $    ; 等待 SBUF 寄存器发送完毕
    CLR TI       ; 清除发送完成标志
    SJMP SEND_STRING ; 继续发送下一个字符

END:
    NOP          ; 程序结束,停在此处

MSG:
    DB 'Hello, World!', 0  ; 字符串数据,以 0 结束
\end{lstlisting}

\subsection{C 语言代码示例}
\begin{lstlisting}[language=C]
#include <stdio.h>

int main() {
    printf("Hello, World!\n");
    return 0;
}
\end{lstlisting}
%论文为虚构,仅为展示使用%%%%%%%%%%%%%
\clearpage\addcontentsline{toc}{section}{参考文献} % 添加到目录
\bibliography{references} % 加载 .bib 文件
%%%%%%%%%%%%%%%%不用改动

\nocite{*}      % 显示所有参考文献,即使未被引用,如果你只想引用时显示,请删除


%%%%%%附录

\appendix       %开始附录部分
\phantomsection % 手动添加锚点,确保超链接正确

\section{第一个附录}
这是附录 A 的内容。

\begin{figure}[htbp]
  \centering
  \includegraphics[width=0.5\textwidth]{example-image}
  \caption{示例图片}
  \label{fig:A1}
\end{figure}


\vspace{2cm}
\begin{table}[h] % 使用 [h] 强制表格位置
  \centering
  \caption{原神角色基础数值比较} % 表格标题
  \label{tab:genshin} % 表格标签
  \begin{tabular}{lcccc} % 定义表格列格式(5 列,第一列左对齐,其余居中)
    \toprule % 顶部横线
    \textbf{角色} & \textbf{生命值} & \textbf{攻击力} & \textbf{防御力} & \textbf{元素精通} \\
    \midrule % 中间横线
    刻晴 & 13103 & 323 & 799 & 96 \\
    胡桃 & 15552 & 106 & 876 & 0 \\
    钟离 & 14695 & 251 & 738 & 0 \\
    \multirow{2}{*}{雷电将军} & 12907 & 337 & 789 & 0 \\ % 合并单元格
                              & (初始) & (初始) & (初始) & (初始) \\
    \bottomrule % 底部横线
  \end{tabular}
\end{table}



\section{第二个附录}
这是附录 B 的内容。
\begin{equation} \label{eq1}
\int e^{-x^2} \, dx = \sqrt{\pi}
\end{equation}

\begin{equation} \label{eq2}
\frac{d}{dx} \int f(t) \, dt = f(x)
\end{equation}

\begin{equation} \label{eq3}
\nabla \cdot (\nabla \times \mathbf{F}) = 0
\end{equation}

\begin{equation} \label{eq4}
\det(A) = \prod \lambda_i
\end{equation}

\begin{equation} \label{eq5}
A \mathbf{x} = \lambda \mathbf{x}
\end{equation}

\begin{equation} \label{eq6}
\sum_{k=1}^n k = \frac{n(n+1)}{2}
\end{equation}

\begin{equation} \label{eq7}
\lim_{n \to \infty} \left(1 + \frac{1}{n}\right)^n = e
\end{equation}

\begin{equation} \label{eq8}
\frac{\partial^2 u}{\partial t^2} = c^2 \frac{\partial^2 u}{\partial x^2}
\end{equation}

\begin{equation} \label{eq9}
\oint \mathbf{F} \cdot d\mathbf{r} = \iint (\nabla \times \mathbf{F}) \cdot d\mathbf{S}
\end{equation}

\begin{equation} \label{eq10}
\mathcal{L}\{f(t)\} = \int_0^\infty e^{-st} f(t) \, dt
\end{equation}

\begin{equation} \label{eq11}
\mathbb{E}[X] = \int_{-\infty}^\infty x f(x) \, dx
\end{equation}

\begin{equation} \label{eq12}
\text{Var}(X) = \mathbb{E}[X^2] - (\mathbb{E}[X])^2
\end{equation}

\begin{equation} \label{eq13}
P(A|B) = \frac{P(B|A) P(A)}{P(B)}
\end{equation}

\begin{equation} \label{eq14}
J = \frac{\partial(x, y, z)}{\partial(u, v, w)}
\end{equation}

\begin{equation} \label{eq15}
\frac{d^2 y}{dx^2} + p(x) \frac{dy}{dx} + q(x) y = 0
\end{equation}

\begin{equation} \label{eq16}
A = P D P^{-1}
\end{equation}

\begin{equation} \label{eq17}
\int_a^b f(x) \, dx \approx \frac{b-a}{n} \sum_{i=1}^n f\left(a + \frac{(b-a)i}{n}\right)
\end{equation}

\begin{equation} \label{eq18}
\nabla f = \left( \frac{\partial f}{\partial x}, \frac{\partial f}{\partial y}, \frac{\partial f}{\partial z} \right)
\end{equation}

\begin{equation} \label{eq19}
\frac{\partial}{\partial t} \left( \frac{1}{2} \rho v^2 + \rho g h + p \right) = 0
\end{equation}

\begin{equation} \label{eq20}
\mathbf{F} = m \mathbf{a}
\end{equation}

%%%%%%%%%%%%%%%%%致谢
\acknowledgement

我要感谢ChatGPT、deepseek、Claude的帮助,排名不分前后\footnote{基于深度学习的智能对话模型。},尤其是Deepseek.它的效果出乎我的意料,帮助我解决了很多难题,如果没有deepseek等等ai agent的帮助,对于我来说,这个模板的实现将遥不可及,是ChatGPT、deepseek、和Claude等等人工智能助我一臂之力,在此奉上我最诚挚的感谢,其次我要感谢SCNU的同学的论文原始模板,是它给了我模板灵感与基础架构,最后我要感谢自己,今天少玩了几把王者荣耀,把这个模板做出来,虽然还有很多不足,比如说字体的设置与官方的word文档有些不一样,参考文献添加了序号,目录做了空格处理,以及附录的页边距问题等等,这些问题还是等待有缘人来解决吧。

还有一个问题就是在附录AB时因为需要根据页码来做AB序号,所以重置页码数,这导致了附录A B无法正确超链接转到该页,我百思不得其解。


再次感谢\textbf{$deepseek$},在众多大模型中帮助我最多,还很少出现幻觉的那个。

谢谢你,\textbf{deepseek V3}!


\hfill January 1, 2025

\hfill 残橘子














\end{document}
